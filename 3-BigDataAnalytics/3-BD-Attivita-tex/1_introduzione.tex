Per molti anni i database relazionali sono stati la principale soluzione utilizzata dalla maggior parte dei sistemi per il salvataggio persistente dei dati e per la loro gestione.

Con la diffusione di servizi \textit{web-based} e le relativa gestione di grandi moli di dati, questo tipo di soluzioni ha iniziato ad evidenziare dei limiti strutturali. La difficoltà di questi sistemi a scalare con la quantità di dati in termini di performance nelle operazioni svolte hanno portato alla ricerca di soluzioni alternative per lo \textit{storage} dei dati.

In questo contesto si sono diffusi i primi database NoSQL e in particolare quelli di tipologia \textit{key-value}, che cercano di soddisfare questa richiesta di alte performance con la semplicità nell'accesso ai dati, un'alta scalabilità e con la distribuzione dei dati su diversi \textit{data storage}.

In questa relazione vengono approfondite le caratteristiche principali di questo modello NoSQL, facendo riferimento ad alcune sue implementazioni, in particolare a DynamoDB di Amazon. Dynamo è stato il primo tra i sistemi che implementano questo modello e ha introdotto le caratteristiche e le tecniche che sono state poi riprese dalle successive implementazioni.