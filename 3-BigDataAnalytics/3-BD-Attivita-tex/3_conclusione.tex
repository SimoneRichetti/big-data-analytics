Il punto di forza dei database \textit{key-value} risiede nelle alte performance dovute alla loro semplicità. L'accesso in memoria è diretto, senza un \textit{query language} e senza proiezioni o \textit{filtering} dei dati, e i dati possono essere distribuiti su più storage con un design resistente a eventuali guasti della rete, risultando in un'immediata disponibilità dei dati e in un sistema scalabile. Queste caratteristiche rendono il modello chiave-valore ideale per sistemi \textit{always-on}, con forti requisiti di performance e che devono gestire grandi moli di dati. Per questo motivo, i database \textit{key-value} sono ampiamente utilizzati in casi d'uso come la memorizzazione di sessioni utente in applicazioni web, la gestione del carrello in grandi sistemi \textit{e-commerce} tra cui Amazon, gestione di profili utente e delle loro preferenze o sistemi di \textit{recommendation} di prodotti.

% For most key-value stores, the secret to its speed lies in its simplicity. The path to retrieve data is a direct request to the object in memory or on disk. The relationship between data does not have to be calculated by a query language; there is no optimization performed. They can exist on distributed systems and don’t need to worry about where to store indexes, how much data exists on each system or the speed of a network within a distributed system they just work.

% Some key-value stores like Aerospike, take advantage of other attributes to extend performance, such as using SSD’s or flash storage and implementing secondary indexes to continue to push the limits of today’s technology to places we’ve not yet conceived.

% %=====

% Generally speaking, the secret to key-value databases lies in their simplicity and the resulting speed that becomes available. Retrieving data requires a direct request (key) for the object in memory (value), and there is no query language. The data can be stored on distributed systems with no worries about where indexes are located, the volume of data, or network slowdowns. Some key-value databases are using flash storage and secondary indexes in an effort to push the limits of key-value technology.

% A key-value database is both easy to build and to scale. It typically offers excellent performance and can be optimized to fit an organization’s needs. When a key-value database is modified with new applications, there is an increased chance the system will operate more slowly.
